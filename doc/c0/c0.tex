\documentclass[a4paper]{article}
% \usepackage[a4paper]{geometry}
\usepackage{fullpage}
\usepackage{times}
\usepackage{epsfig,endnotes}
\usepackage{algorithm}
\usepackage{algorithmic}
\usepackage{amssymb,amsmath}
\usepackage{color}
\usepackage{comment}
\usepackage{multirow}
\usepackage{graphicx,subfigure}
\usepackage{url}
\usepackage{alltt}
\usepackage{listings}
\usepackage[normalem]{ulem}
\lstset{basicstyle=\ttfamily\small, language=C, breaklines=true}


\begin{document}

\begin{flushleft}
\Large \textbf{C0}
\line(1, 0){450}
\end{flushleft}

\section{Introduction}
C0 is grammatically similar to the C language and will be immediately familiar to C, C++ and Java programmers. It is a procedure-oriented language with linguistic support for massive parallelism on a modern compute cluster. 

\subsection{Simple example}
Here we have a parallel version of vector addition it in C0.
\lstinputlisting[language=C]{examples/add.c}
The above program adds two vectors of length 10000 with 100 runners, each runner adds up 100 elements. A runner is a separate execution of code which is similar to threads.

\subsection{Program structure}
The four key concepts in C0 are programs, types, variables and functions. A program consists of one or more source file. Each source file defines some types or functions. The program must have a function named main with no parameter or return value. The main function is where the program starts.

\subsection{Keywords}
Note: The key words add yellow are not supported currently.

\begin{table}[htbp]
\centering
\caption{C0 key words}
\begin{tabular}{|l|l|l|l|}
\hline
abort & default & goto & static\\
\hline
auto & continue & if & struct\\
\hline
bool & double & int & switch\\
\hline
break & else & long & true\\
\hline
case & enum & return & unsigned\\
\hline
char & extern & runner & void\\
\hline
commit & commitd & false & volatile\\
\hline
const & float & signed & watching\\
\hline
while & for & do & register\\
\hline
in & sizeof & short & standalone\\
\hline
\end{tabular}
\label{table:key-words}
\end{table}

\section{Types}
There are several types in C0: simple types, struct types, union types, function types, void type, pointer types, array types, and array segments.

\subsection{Simple types}
Table~\ref{table:c0-types} shows the simple types supported (Or would be supported) in C0.

\begin{table}[htbp]
\centering
\caption{Simple types in C0}
\begin{tabular}{|l|l|l|l|}
\hline
category & bits & type & range/precision\\
\hline
boolean & 32 & bool & true or false\\
\hline
\multirow{4}{*}{signed integral} & 8 & char & -128...127\\
 & 16 & - & –32,768...32,767\\
 & 32 & int & –2,147,483,648...2,147,483,647\\
 & 64 & long & –9,223,372,036,854,775,808...9,223,372,036,854,775,807\\ \hline
\multirow{4}{*}{unsigned integral} & 8 & unsigned char & 0...255\\
 & 16 & - & 0...65,535\\
 & 32 & unsigned int & 0...4,294,967,295\\
 & 64 & unsigned long & 0...18,446,744,073,709,551,615\\ \hline
\multirow{2}{*}{floating point} & 32 & float & 1.5 × 10−45 to 3.4 × 1038, 7-digit precision\\
 & 64 & double & 5.0 × 10−324 to 1.7 × 10308, 15-digit precision\\ \hline
\end{tabular}
\label{table:c0-types}
\end{table}

\subsection{Struct/Union types (not supported yet)}
Structure types are user defined types which contains other types (including other structure types).  The struct keyword is used to define a structure type. Each element of a structure is called field. Each field in a structure has its own storage space.
\lstinputlisting[language=C]{examples/struct.c}
The union types are similar to structure types. But the field in union shares the common storage space, so at most one field contains a meaningful value at any given time.

\subsection{Function types}
In the program, you cannot directly define variables of function types. But you can define functions who has a function type, or define a function pointer to a specified function type.
A function type describes the function prototype, including the types of parameters and the type of return value.

\subsection{Void type}
Void type is a special type which means "no type", it can only be used for the return type of function, which means the function does not return any value, or used for defining a pointer which can points to any kind of values.

\subsection{Pointer types}
A variable of pointer type stores the address of the underlying type. We can access the value stored in the memory location which the pointer points to. This operation is called dereferencing a pointer. However, a pointer whose underlying type is void type cannot be dereferenced. 

\subsection{Array types}
An array is a data structure that contains a number of variables that are accessed through computed indices. The variables contained in an array, also called the elements of the array, are all of the same type, and this type is called the element type of the array. We use array[index] to access the elements of an array. The indices of the elements of an array range from 0 to Length - 1. 

\subsection{Array segment}
An array segment is logically same as an array (or a pointer). However, it restricts the access of elements to a specified range. The array segment is represented as array[start,,end], the start is inclusive and end is exclusive.

\subsection{Standalone}
standalone is a special keyword of C0. It is for global variables which are in the Shared Region (SR). When a global variable is standalone, it always monopolizes one or many memory pages. The main purpose of using standalone is to reduce commit conflict among different global variables, since the basic unit of memory space management is a page.

\begin{verbatim}
standalone long a;
standalone long b;
\end{verbatim}
For the above example, if runner A modify a and runner B modify b, there will be no commit conflict.

\section{Expressions and Statements}

\subsection{Expressions}
Expressions are constructed from operands and operators. The operators of an expression indicate which operations to apply to the operands. Examples of operators include +, -, *, /. Examples of operands include literals, fields, local variables, and expressions.
When an expression contains multiple operators, the precedence of the operators controls the order in which the individual operators are evaluated. For example, the expression x + y * z is evaluated as x + (y * z) because the * operator has higher precedence than the + operator.
Table~\ref{table:c0-operators} summarizes C0 operators, listing the operator categories in order of precedence from highest to lowest. Operators in the same category have equal precedence.

\begin{table}[htbp]
\centering
\caption{Operators in C0}
\begin{tabular}{|l|l|l|}
\hline
Category & Expression & Description\\
\hline
\multirow{6}{*}{Primary} & x.m & Field access\\
 & x(...) & Method invocation\\
 & x[...] & Array or array segment access\\
 & x++ & Post-increment\\
 & x-- & Post-decrement\\
 & x->y & Pointer\\ \hline
\multirow{9}{*}{Unary} & *x & Dereference\\
 & \&x & Referencing the address\\
 & +x & Identity\\
 & -x & Negation\\
 & !x & Logical negation\\
 & ~x & Bitwise negation\\
 & ++x & Pre-increment\\
 & --x & Pre-decrement\\
 & (T)x & Explicitly convert x to type T\\ \hline
\multirow{3}{*}{Multiplicative} & x * y & Multiplication\\
 & x / y & Division\\
 & x \% y & Remainder\\ \hline
\multirow{2}{*}{Additive} & x + y & Addition\\
 & x - y & Subtraction\\ \hline
\multirow{2}{*}{Shift} & x << y & Shift left\\
 & x >> y & Shift right\\ \hline
\multirow{4}{*}{Relational} & x < y & Less than\\
 & x > y & Greater than\\
 & x <= y & Less than or equal\\
 & x >= y & Greater than or equal\\ \hline
\multirow{2}{*}{Equality} & x == y & Equal\\
 & x != y & Not equal\\ \hline
Bitwise AND & x \& y & Integer bitwise AND\\ \hline
Bitwise XOR & x \^ y & Integer bitwise XOR\\ \hline
Bitwise OR & x | y & Integer bitwise OR\\ \hline
Logical AND & x \&\& y & Boolean logical AND\\ \hline
Logical OR & x || y & Boolean logical OR\\ \hline
Conditional & x ? y : z & Evaluates y if x is true, z if x is false\\ \hline
\multirow{2}{*}{Assignment} & x = y & Assignment\\
 & x op= y & Compound assignment; supported operators are *=  /= \%= +=  -=  <<=  >>=  \&=  \^=  |=\\ \hline
\end{tabular}
\label{table:c0-operators}
\end{table}

\subsection{Statements}
The actions of a program are expressed using statements. 
A block permits multiple statements to be written in contexts where a single statement is allowed. A block consists of a list of statements written between the delimiters { and }.
Declaration statements are used to declare local variables and constants.
Expression statements are used to evaluate expressions. Expressions that can be used as statements include method invocations, assignments using = and the compound assignment operators, and increment and decrement operations using the ++ (Not supported yet) and -- (Not supported yet) operators.
Selection statements are used to select one of a number of possible statements for execution based on the value of some expressions. In this group are the if and switch statements.
Iteration statements are used to repeatedly execute an embedded statement. In this group are the while, do, and for statements.
Jump statements are used to transfer control. In this group are the break, continue, goto, and return statements.


\section{Task and depending task}

\subsection{Create a runner (task)}

Defining a runner is just the same as defining a function. Actually any function satisfying the necessary constraints (will be mentioned later) can be started as a runner. A same function can either be directly invoked or be started as a new runner.
The function that can become a runner must have the prototype with the following constraints

\begin{itemize}
	\item It has no return type (with return type void)
	\item The parameters can only be either 1) simple types, or 3) array segments, or 3) structure types whose fields meet the constraints of 1) or 3).
\end{itemize}

The above constraints ensure that the input parameters to a new runner will not reference external memory locations not in the range of the parameters. The use of array segments constraints the use of pointers so the runtime can create the snapshots efficiently.

\subsection{Creating instance of runners}

The syntax of creating a runner is the same as invoking a function, plus the keyword runner. Note that the runner will only start to execute after current runner exits.
Example (quick sort):

\lstinputlisting[language=C]{examples/sort.c}

\subsection{Watchers (depending tasks)}

The watchers are runners with additional startup conditions. Specifically, it will start after the parent runner commits successfully and the specified memory location has been modified since the creation of the watcher.
Defining a watcher is similar to define a normal runner.
To create a watcher, we also use runner keywords, with additional parameters to specify the memory location to watch. The watcher will get executed if the content of the memory has changed. The parameter can either be the pointer to a simple type or structure type, or an array segment.

\lstinputlisting[language=C]{examples/watcher.c}

\subsection{Creating runners in another space}

The memory space in i0 is separated into many spaces which is a continuous range. Each space has a space specifier and the offset ranges for all spaces are the same. By default, the runner statement creates runners in the same space as the parent runner. The space can be specified by the in clause of the runner statement.

For example, to create a qsort runner in space SPACE1:

\lstinputlisting[language=C]{examples/runner.c}

\section{Runtime Environment}

\subsection{Memory Layout}
Each runner runs in one space of the many spaces supported. The AMR is accessible from all spaces and can be used to share data across spaces. This following figure illustrates the spaces.



\end{document}
